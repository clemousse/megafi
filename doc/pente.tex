% © 2015 (November 1st) alkra
% alban.kraus@ensg.eu

\documentclass[a4paper, 11pt, twocolumn]{article}

\usepackage[utf8]{inputenc}
\usepackage[T1]{fontenc}
\usepackage[french]{babel}
\usepackage{lmodern}

\usepackage{array}
\usepackage{tikz}

\makeatletter
\newcommand\@point[2]{\fill[#1] #2 circle [radius=2pt]}
\newcommand\point[4][black]{%
  \def\pt{#2}
  \ifx \pt\@arrivee
  \@point{red}{(#2)}%
  \else
  \@point{#1}{(#2)}
  \fi
  node[black, anchor=#3]{#4};
  
  \ifnum \@min > #4
  \@min=#4
  \global\def\@argmin{#2}
  \fi
}
\newcommand\@carre[6]{
  \coordinate (a) at (0,1);
  \coordinate (b) at (1,1);
  \coordinate (c) at (1,0);
  \coordinate (d) at (0,0);

  \def\@arrivee{#5}
  \global\newcount\@min
  \@min=#1
  \global\def\@argmin{a}

  \point[blue]{a}{south east}{#1}
  \point{b}{south west}{#2}
  \point{c}{north west}{#3}
  \point{d}{north east}{#4}

  \draw (a) -- (b) -- (c) -- (d) -- cycle;  

  #6

  \draw[%
  \ifx \@argmin\@arrivee
  green%
  \else
  red%
  \fi
  , thick] (a) -- (\@arrivee);
}
\newcommand\carre[6]{%
  \begin{tikzpicture}
    \@carre{#1}{#2}{#3}{#4}{#5}{%
      \draw (a) -- (c);%
    }
  \end{tikzpicture} &
  \begin{tikzpicture}
    \@carre{#1}{#2}{#3}{#4}{#6}{%
      \draw (b) -- (d);%
    }
  \end{tikzpicture}
}
\makeatother

\newlength\colonneD
\setlength\colonneD{\linewidth}
\addtolength\colonneD{-2.5cm}

\begin{document}

\section*{0 plus petits}

\begin{tabular}{p{0.45\linewidth} p{0.45\linewidth}}
  \carre{5}{8}{9}{8}{a}{a}
\end{tabular}

\section*{1 plus petit}

\begin{tabular}{p{0.45\linewidth} p{0.45\linewidth}}
  \carre{5}{3}{9}{8}{b}{b}\\
  \carre{5}{8}{3}{8}{c}{a}
\end{tabular}

\section*{2 plus petits}

\begin{tabular}{p{0.45\linewidth} p{0.45\linewidth}}
  \carre{5}{3}{9}{2}{d}{d}\\
  \carre{5}{8}{3}{2}{d}{d}\\
  \carre{5}{8}{2}{3}{c}{d}
\end{tabular}

\section*{3 plus petits}

Idem qu'avec deux plus petits

\section*{Conclusion}

Raisonner par triangle

\end{document}
